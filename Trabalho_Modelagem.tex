\documentclass[11pt,a4paper]{article}
\usepackage[utf8]{inputenc}
\usepackage[T1]{fontenc}
\usepackage{blindtext}
\usepackage{enumitem}
\usepackage{hyperref}

\begin{document}


\section{Informações Estudante}
\begin{itemize}
\item Aluno: João Henrique Pereira de Almeida
\item Matrícula: 15/0132042
\item Perfil Github: \href{http://github.com/joao-henrique/}{http://github.com/joao-henrique/}
\item email: joaohenrique.p.almeida@gmail.com
\item Disciplina: Fundamentos de Sistemas Operacionais

\end{itemize}


\section{Informações Importantes}
\subsection{Repositório}
O codigo dessa atividade se encontra no repositório abaixo, sob licença MIT
\url{https://github.com/joao-henrique/FSO-2016.1/tree/master/trab01}
\subsection{Sistema Operacional}


Foi usado para o desenvolvimento das atividades propostas o ElementaryOS - Loki, sistema operacional baseado no Ubuntu 16.04

\subsection{Ambiente de Desenvolvimento }
- O código encontrado nos três exercicios funcionaram em
conformidade em ambiente GNU/Linux OS, juntamente com o GCC-5

\section{Instruções}


\subsection{Instruções para compilação}
Aqui será listado os comandos que devem ser executados no terminal, dentro da pasta raiz.

\subsubsection{Questão 01}
No diretório raiz do projeto utilize os seguintes comandos
\begin{verbatim}
$ cd q01/
$ make
\end{verbatim}

\subsubsection{Questão 02}
No diretório raiz do projeto utilize os seguintes comandos
\begin{verbatim}
$ cd q02/
$ make
\end{verbatim}

Após uso desse comando deve-se passar as opções juntamente com os inteiros
para o uso

\begin{verbatim}
Opções
-crescent, 			 "Listar em ordem crescente"
-decreasing, 		 "Listar em ordem decrescente"
-help 		       "Mostrar Exemplo"


Para mostrar exemplo na linha de comando
$ ./run -help

Para listar em ordem crescente
$ ./run -crescent  9 8 7 4 5 6 1 2 3

Para listar em ordem decrescente
$ ./run -decreasing 9 8 7 4 5 6 1 2 3

Como ação default algoritmo executa a ordem crescente quando não é passado nenhuma opção
$ ./run 9 8 7 4 5 6 1 2 3

\end{verbatim}











\subsubsection{Questão 03}
No diretório raiz do projeto utilize os seguintes comandos
\begin{verbatim}
$ cd q03/
$ make
\end{verbatim}

\subsection{Casos de Teste}

\subsubsection{Questão 01}
\begin{itemize}
  \item Na entrada dos vertices do triângulo  não é verificado se a cadeia de
  strings informada são letras ou numeros

  \begin{itemize}
    \item Exemplo 01: Entrada: FSODisciplina, Saída: "The triangle does not exists";
    \item Exemplo 02: Entrada:1,8,5,6,7,9 , \\
    Saída: " The area of the triangle is: 8.00; The side of the vertice n1 is: 4.47; \\
    The side of the vertice n2 is: 3.61;\\
    The side of the vertice n3 is: 6.08;\\
    The perimeter of the triangle is: 14.16;\\
  \end{itemize}

  \item É verificado se os pontos dos vertices informados estão ma mesma reta do plano
  \begin{itemize}
    \item Exemplo: Entrada: 1,2,3,4,5,6, Saída: "The triangle does not exists";
  \end{itemize}
\end{itemize}

\subsubsection{Questão 02}
\begin{itemize}
  \item
  \begin{itemize}
    \item Opção decrescente \\
    Entrada:
    \begin{verbatim}
      $ ./run -decreasing  9 8 7 4 5 6 1 2 3
    \end{verbatim}
    Saída esperada:
    \begin{verbatim}
      9 8 7 6 5 4 3 2 1
    \end{verbatim}
  \end{itemize}

  \begin{itemize}
    \item Opção crescente \\
    Entrada:
    \begin{verbatim}
      $ ./run -crescent  9 8 7 4 5 6 1 2 3
    \end{verbatim}
    Saída esperada:
    \begin{verbatim}
      0 1 2 3 4 5 6 7 8 9
    \end{verbatim}
  \end{itemize}

  \begin{itemize}
    \item Opção default \\
    Entrada:
    \begin{verbatim}
      $ ./run  9 8 7 4 5 6 1 2 3
    \end{verbatim}
    Saída esperada:
    \begin{verbatim}
      1 2 3 4 5 6 7 8 9
    \end{verbatim}
  \end{itemize}

  \item Caso no array de informações
  \begin{itemize}
    \item Opção com entrada com letras \\
    O sistema entende as letras como zero(0) e organiza o restante
    Entrada:
    \begin{verbatim}
      $ ./run  J 8 7 T 5 6 1 2 3
    \end{verbatim}
    Saída esperada:
    \begin{verbatim}
      0 0 1 2 3 5 6 7 8
    \end{verbatim}
  \end{itemize}


  \item Opção de ajuda
  \begin{itemize}
    \item Entrada
    \begin{verbatim}
      $ ./run  -h
    \end{verbatim}
    Saída esperada:
    \begin{verbatim}
    Opções
    -crescent, 			 'Listar em ordem crescente'
    -decreasing, 		 'Listar em ordem decrescente'
    -help 		       'Mostrar Exemplo'Exemplos:
    Example:
      ./run  9 8 7 4 5 6 1 2 3
	ou:   ./run -crescent  9 8 7 4 5 6 1 2 3
	ou:	  ./run -decreasing  9 8 7 4 5 6 1 2 3
	ou:	  ./run -help
\end{verbatim}
\end{itemize}


\item Opção de ajuda sempre terá preferência na execução dos comandos
\begin{itemize}
  \item Entrada
  \begin{verbatim}
    $ ./run -h  9 8 7 4 5 6 1 2 3
  \end{verbatim}
  Saída esperada:
  \begin{verbatim}
  Opções
  -crescent, 			 'Listar em ordem crescente'
  -decreasing, 		 'Listar em ordem decrescente'
  -help 		       'Mostrar Exemplos:
  Example:
    ./run  9 8 7 4 5 6 1 2 3
ou:   ./run -crescent  9 8 7 4 5 6 1 2 3
ou:	  ./run -decreasing  9 8 7 4 5 6 1 2 3
ou:	  ./run -help
\end{verbatim}
\end{itemize}
\end{itemize}

\subsubsection{Questão 03}

Entrada:
\begin{verbatim}
 $ Universidade de Brasilia
\end{verbatim}
Saída esperada:
\begin{verbatim}
 $ Value of dPtr: 7.300000
 $ Value of number2: 7.300000
 $ Pointer to number1: 0x7fff90d0d218
 $ Pointer to dPtr: 0x7fff90d0d218
 $ Put a string: Universidade de Brasilia
 $ Compare to s1 and s2: 0
 $ Concat of s1 and s2: UniversidadeUniversidade
 $ Len of |UniversidadeUniversidade| is |24|
\end{verbatim}

\subsection{Limitações conhecidas}
\subsubsection{Questão 01}
\begin{itemize}
  \item Não há um tratamento dos valores de entrada no sistema
\end{itemize}

\subsubsection{Questão 02}
\begin{itemize}
  \item Não há um tratamento para entradas inválidas no sistema
\end{itemize}

\subsubsection{Questão 03}
\begin{itemize}
\item
O valor impresso decorrente do enunciado que contempla o item anterior é igual ao
valor do endereço gravado em dPrt?
\begin{verbatim}
    Não, é apenas um ponteiro indicando um endereço
    \end{verbatim}

\item
  A execução do item anterior pode provocar algum erro em tempo de execução?
\begin{verbatim}
    Sim, pode ocorrer um problema com a concatenação e o tamanho não suficiente \\
    para armazenar ambas strings
\end{verbatim}



\end{itemize}

\end{document}
