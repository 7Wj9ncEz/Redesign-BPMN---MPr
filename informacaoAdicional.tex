\begin{flushleft}
\section{Informações Adicionais}

\subsection{Motivação}
Quando se pensa sobre os principais fatores que colaboram para o sucesso de
empresas a  satisfação do cliente no mercado de TI exerce um papel importante,
e \textit{user experience} é com certeza um dos principais fatores, os caminhos
para prover a \textit{user experience} são muitos, por exemplo prover qualidade
no serviço de usuário\cite{userQuant}, pode ser um desses caminhos. Consequentemente vemos o
suporte ao usuário como sendo um serviço provido por uma empresa  ao seus
clientes com o objetivo de melhorar a experiência com o produto provido. Em
outras palavras o serviço de suporte ao usuário ajuda ao cliente resolver
qualquer problema que possa encontrar enquanto usa o produto ou serviço\cite{reaganUsing}

\end{flushleft}

\subsubsection{Serviço de suporte ao usuário}
Primeiramente devemos definir o que é o serviço de suporte na área de TI,
podemos encontrar vários termos como:
\begin{itemize}[noitemsep]
  \item Suporte tecnico;
  \item Service Desk;
  \item Help Desk;
  \item Suporte ao Cliente;
  \item Suporte;
  \item Suporte ao usuário;
  \item Etc.
\end{itemize}

Basicamente esses termos definem a mesma coisa, mas cada um deles é focado
em diferentes aspectos do serviço, então para isso vamos focar no suporte
ao usuário, pois esse é mais comum e com isso evitamos ambiguidades.
Logo entende-se como suporte ao usuário um serviço provido por uma organização
para seus clientes para promover uma experiência com seu produto ou serviço,
resolvendo qualquer problema que o cliente possa encontrar enquanto usa o
serviço ou produto\cite{Hassenzahl}. Além disso não se  deve colocar qualquer restrição ao tipo
de problema que possa ser encontrado ou qualquer dúvida ou denuncia que o
cliente possa reportar.
